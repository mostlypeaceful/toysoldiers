%%%%%%%%%%%%%%%%%%%%%%%%%%%%%%%%%%%%%%%%%%%%%%%%%%%%%%%%%%%%%%%%%%%%%%%%%%%%%%%
%% Name:        animation.tex
%% Purpose:     wxAnimation documentation
%% Author:      Francesco Montorsi
%% Created:     24-9-2006
%% RCS-ID:      $Id: animation.tex 43898 2006-12-10 14:18:37Z VZ $
%% Copyright:   (c) 2006 Francesco Montorsi
%% License:     wxWindows license
%%%%%%%%%%%%%%%%%%%%%%%%%%%%%%%%%%%%%%%%%%%%%%%%%%%%%%%%%%%%%%%%%%%%%%%%%%%%%%%

\section{\class{wxAnimation}}\label{wxanimation}

This class encapsulates the concept of a platform-dependent animation.
An animation is a sequence of frames of the same size.
Sound is not supported by wxAnimation.

\wxheading{Derived from}

\helpref{wxGDIObject}{wxgdiobject}\\
\helpref{wxObject}{wxobject}

\wxheading{Include files}

<wx/animate.h>

\wxheading{Predefined objects}

Objects:

{\bf wxNullAnimation}

\wxheading{See also}

\helpref{wxAnimationCtrl}{wxanimationctrl}

\latexignore{\rtfignore{\wxheading{Members}}}


\membersection{wxAnimation::wxAnimation}\label{wxanimationctor}

\func{}{wxAnimation}{\void}

Default constructor.

\func{}{wxAnimation}{\param{const wxAnimation\& }{anim}}

Copy constructor, uses \helpref{reference counting}{trefcount}.

\func{}{wxAnimation}{\param{const wxString\& }{name}, \param{wxAnimationType}{ type = wxANIMATION\_TYPE\_ANY}}

Loads an animation from a file.

\docparam{name}{The name of the file to load.}

\docparam{type}{See \helpref{LoadFile}{wxanimationloadfile} for more info.}


\membersection{wxAnimation::\destruct{wxAnimation}}\label{wxanimationdtor}

\func{}{\destruct{wxAnimation}}{\void}

Destructor.
See \helpref{reference-counted object destruction}{refcountdestruct} for more info.


\membersection{wxAnimation::GetDelay}\label{wxanimationgetdelay}

\constfunc{int}{GetDelay}{\param{unsigned int }{i}}

Returns the delay for the i-th frame in milliseconds.
If {\tt -1} is returned the frame is to be displayed forever.


\membersection{wxAnimation::GetFrameCount}\label{wxanimationgetframecount}

\constfunc{unsigned int}{GetFrameCount}{\void}

Returns the number of frames for this animation.


\membersection{wxAnimation::GetFrame}\label{wxanimationgetframe}

\constfunc{wxImage}{GetFrame}{\param{unsigned int }{i}}

Returns the i-th frame as a \helpref{wxImage}{wximage}.


\membersection{wxAnimation::GetSize}\label{wxanimationgetsize}

\constfunc{wxSize}{GetSize}{\void}

Returns the size of the animation.


\membersection{wxAnimation::IsOk}\label{wxanimationisok}

\constfunc{bool}{IsOk}{\void}

Returns \true if animation data is present.


\membersection{wxAnimation::Load}\label{wxanimationload}

\func{bool}{Load}{\param{wxInputStream\&}{ stream}, \param{wxAnimationType}{ type = wxANIMATION\_TYPE\_ANY}}

Loads an animation from the given stream.

\wxheading{Parameters}

\docparam{stream}{The stream to use to load the animation.}

\docparam{type}{One of the following values:

\twocolwidtha{5cm}
\begin{twocollist}
\twocolitem{\indexit{wxANIMATION\_TYPE\_GIF}}{Load an animated GIF file.}
\twocolitem{\indexit{wxANIMATION\_TYPE\_ANI}}{Load an ANI file.}
\twocolitem{\indexit{wxANIMATION\_TYPE\_ANY}}{Try to autodetect the filetype.}
\end{twocollist}
}

\wxheading{Return value}

\true if the operation succeeded, \false otherwise.


\membersection{wxAnimation::LoadFile}\label{wxanimationloadfile}

\func{bool}{LoadFile}{\param{const wxString\&}{ name}, \param{wxAnimationType}{ type = wxANIMATION\_TYPE\_ANY}}

Loads an animation from a file.

\wxheading{Parameters}

\docparam{name}{A filename.}

\docparam{type}{One of the following values:

\twocolwidtha{5cm}
\begin{twocollist}
\twocolitem{\indexit{wxANIMATION\_TYPE\_GIF}}{Load an animated GIF file.}
\twocolitem{\indexit{wxANIMATION\_TYPE\_ANI}}{Load an ANI file.}
\twocolitem{\indexit{wxANIMATION\_TYPE\_ANY}}{Try to autodetect the filetype.}
\end{twocollist}
}

\wxheading{Return value}

\true if the operation succeeded, \false otherwise.


\membersection{wxAnimation::operator $=$}\label{wxanimationassignment}

\func{wxAnimation\&}{operator $=$}{\param{const wxAnimation\& }{brush}}

Assignment operator, using \helpref{reference counting}{trefcount}.

