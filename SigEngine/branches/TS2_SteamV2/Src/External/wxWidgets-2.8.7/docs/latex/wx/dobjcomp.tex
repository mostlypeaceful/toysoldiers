%%%%%%%%%%%%%%%%%%%%%%%%%%%%%%%%%%%%%%%%%%%%%%%%%%%%%%%%%%%%%%%%%%%%%%%%%%%%%%%
%% Name:        dobjcomp.tex
%% Purpose:     wxDataObjectComposite documentation
%% Author:      Vadim Zeitlin
%% Modified by:
%% Created:     02.11.99
%% RCS-ID:      $Id: dobjcomp.tex 38736 2006-04-15 05:35:08Z RD $
%% Copyright:   (c) Vadim Zeitlin
%% License:     wxWindows license
%%%%%%%%%%%%%%%%%%%%%%%%%%%%%%%%%%%%%%%%%%%%%%%%%%%%%%%%%%%%%%%%%%%%%%%%%%%%%%%

\section{\class{wxDataObjectComposite}}\label{wxdataobjectcomposite}

wxDataObjectComposite is the simplest 
\helpref{wxDataObject}{wxdataobject} derivation which may be used to support
multiple formats. It contains several 
\helpref{wxDataObjectSimple}{wxdataobjectsimple} objects and supports any
format supported by at least one of them. Only one of these data objects is
{\it preferred} (the first one if not explicitly changed by using the second
parameter of \helpref{Add}{wxdataobjectcompositeadd}) and its format determines
the preferred format of the composite data object as well.

See \helpref{wxDataObject}{wxdataobject} documentation for the reasons why you
might prefer to use wxDataObject directly instead of wxDataObjectComposite for
efficiency reasons.

\wxheading{Virtual functions to override}

None, this class should be used directly.

\wxheading{Derived from}

\helpref{wxDataObject}{wxdataobject}

\wxheading{Include files}

<wx/dataobj.h>

\wxheading{See also}

\helpref{Clipboard and drag and drop overview}{wxdndoverview}, 
\helpref{wxDataObject}{wxdataobject}, 
\helpref{wxDataObjectSimple}{wxdataobjectsimple}, 
\helpref{wxFileDataObject}{wxfiledataobject}, 
\helpref{wxTextDataObject}{wxtextdataobject}, 
\helpref{wxBitmapDataObject}{wxbitmapdataobject}

\latexignore{\rtfignore{\wxheading{Members}}}

\membersection{wxDataObjectComposite::wxDataObjectComposite}\label{wxdataobjectcompositewxdataobjectcomposite}

\func{}{wxDataObjectComposite}{\void}

The default constructor.

\membersection{wxDataObjectComposite::Add}\label{wxdataobjectcompositeadd}

\func{void}{Add}{
    \param{wxDataObjectSimple }{*dataObject},
    \param{bool }{preferred = false}}

Adds the {\it dataObject} to the list of supported objects and it becomes the
preferred object if {\it preferred} is true.


\membersection{wxDataObjectComposite::GetReceivedFormat}\label{wxdataobjectcompositegetreceivedformat}

\constfunc{wxDataFormat}{GetReceivedFormat}{}

Report the format passed to the SetData method.  This should be the
format of the data object within the composite that recieved data from
the clipboard or the DnD operation.  You can use this method to find
out what kind of data object was recieved.

